\begin{itemize}[noitemsep]

\item Burnside’s Lemma\\
- Formula\\
For a group $G=(X,A)$ with set $X$ and action $A$,  
$\vert A\vert\vert X/A \vert = \sum(\vert \text{Fixed points of a}\vert,\ \text{for all } a \in A)$\\
$X/A$ is the set of equivalence classes (orbits) obtained by grouping elements of $X$ that can be transformed into each other by the action.\\
- Explanation\\
Orbit: For a group and an action $f$, connect $a,b$ with an edge if $f(a)=b$; each connected component is an orbit.\\
Number of orbits $=$ sum of (number of fixed points of each action $g$) divided by the number of actions.\\
- Degrees of freedom cheat sheet\\
$n$ rotations: fixed points of rotation $i$ = $\gcd(n,i)$\\
$n$ odd reflections: $(n+1)/2$ symmetry axes (fixed points)\\
$n$ even reflections: $n/2$ axes through vertices (fixed points $n/2+1$) + $n/2$ axes through edges (fixed points $n/2$)

\item Algorithmic Games\\
- Nim Game (last to take wins): XOR $=0 \Rightarrow$ second player wins, otherwise first player wins.\\
- Subtraction Game: if one can take up to $k$ stones per turn, compute each pile mod $(k+1)$ and XOR the results.\\
- Index-$k$ Nim: one can choose up to $k$ piles and remove any number from each; for each binary digit, sum bits across piles and take mod $(k+1)$; if all digits $\equiv 0$, second player wins, otherwise first wins.\\
- Misère Nim: if all piles contain 1 stone $\Rightarrow$ odd $N$ → second wins; otherwise, same rule as normal Nim (XOR $=0$ → second wins).

\item Pick’s Theorem\\
For a simple polygon with lattice vertices:  
$A=I+\frac{B}{2}-1$,  
where $I$ = number of interior lattice points, $B$ = number of boundary lattice points, $A$ = area.

\item Hall’s Marriage Theorem\\
In a bipartite graph $(L,R)$, a perfect matching covering all $L$ exists iff for every subset $S \subseteq L$,  
$\vert S\vert \le \vert N(S)\vert$, where $N(S)$ is the set of neighbors of $S$ in $R$.

\item Simpson’s Rule (Integration)\\
$S_n(f) = \frac{h}{3}[f(x_0)+f(x_n)+ 4\sum f(x_{2i+1}) + 2\sum f(x_{2i})]$\\
If $M = \max \vert f^{(4)}(x) \vert$, then the error bound is  
$E_n \leq \frac{M(b-a)}{180}h^4$.

\item Brahmagupta’s Formula\\
For a cyclic quadrilateral with side lengths $a,b,c,d$:  
$S=\sqrt{(s-a)(s-b)(s-c)(s-d)}$, where $s=(a+b+c+d)/2$.

\item Bretschneider’s Formula\\
For any quadrilateral with sides $a,b,c,d$ and sum of opposite angles $=2\theta$:  
$S=\sqrt{(s-a)(s-b)(s-c)(s-d)-abcd\times \cos^2 \theta}$.

\item Fermat Point\\
The point minimizing the sum of distances to the three vertices of a triangle.\\
If one angle $\ge 120^\circ$, that vertex is the Fermat point. Otherwise, construct equilateral triangles on each side, connect the new vertices to the opposite original vertices — the intersection is the Fermat point.\\
If all angles $< 120^\circ$, minimal sum of distances $=\sqrt{(a^2 + b^2 + c^2 + 4\sqrt 3 S) / 2}$, where $S$ is the area.

\item Euler’s Theorem\\
For coprime integers $a,n$: $a^{\phi(n)}\equiv 1 \pmod n$\\
For all integers: $a^n \equiv a^{n-\phi(n)} \pmod n$\\
If $m\geq \log_2 n$, then $a^m\equiv a^{m\%\phi(n)+\phi(n)}\pmod n$.

\item $g^0+g^1+g^2+\cdots g^{p-2}\equiv -1 \pmod p$ iff $g=1$, otherwise $0$.

\item If $n \equiv 0 \pmod 2$, then $1^n + 2^n + \cdots + (n-1)^n \equiv 0 \pmod n$.

\item Eulerian Numbers\\
Number of permutations $\pi \in S_n$ in which exactly $k$ elements are greater than the previous one.\\
$E(n,k) = (n-k)E(n-1,k-1) + (k+1)E(n-1,k)$\\
$E(n,0) = E(n,n-1) = 1$\\
$E(n,k) = \sum_{j=0}^k(-1)^j\binom{n+1}{j}(k+1-j)^n$

\item Pythagorean Triple\\
Primitive triples $(a,b,c)$ satisfying $a^2+b^2=c^2$ are generated by  
$(a, b, c) = (st, \frac{s^2-t^2}{2}, \frac{s^2+t^2}{2})$, where $\gcd(s,t)=1$, $s>t$.

\end{itemize}
