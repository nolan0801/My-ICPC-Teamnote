\begin{itemize}[noitemsep]
    \item \textbf{Game on a Graph}:  
    A token starts at vertex $s$.  
    Players alternately move the token to an adjacent vertex; if a player cannot move, they lose.\\
    $\Leftrightarrow$ The second player wins if and only if there exists a maximum matching that does not include $s$.

    \item \textbf{Chinese Postman Problem}:  
    Find a minimum-weight walk that visits every edge at least once.\\
    Run Floyd–Warshall to get all-pairs shortest paths, then collect all odd-degree vertices and find a minimum-weight perfect matching among them. (The number of odd vertices is always even.)

    \item \textbf{Unweighted Edge Cover}:  
    Find the smallest (minimum cardinality) set of edges that covers all vertices.\\
    Result: $\vert V\vert - \vert M\vert$, where $M$ is a maximum matching.  
    There are no paths of length 3; the structure consists of multiple star graphs.

    \item \textbf{Weighted Edge Cover}:  
    $\displaystyle \sum_{v \in V} w(v) - \sum_{(u,v) \in M} \big(w(u) + w(v) - d(u,v)\big)$,  
    where $w(x)$ is the minimum weight of an edge incident to vertex $x$.

    \item \textbf{NEERC'18 B}:  
    Each machine requires two workers to operate.\\
    For each machine, create two vertices and connect them with an edge;  
    the answer is $\vert M\vert - \vert\text{machines}\vert$.  
    It helps to think of each edge as contributing $1/2$ to the answer.

    \item \textbf{Minimum Disjoint Cycle Cover}:  
    Find a set of vertex-disjoint cycles (each of length $\ge 3$) covering all vertices.\\
    Each vertex must be incident to exactly two different edges.  
    While this might seem expressible as a flow, edges with capacity 2 can only carry 1 unit of flow — so a standard flow model fails.\\
    Instead, duplicate every vertex and edge (e.g. $(v, v')$, $(e_{i,u}, e_{i,v})$).  
    For each edge $e=(u,v)$, connect $e_u$ and $e_v$ with an edge of weight $w$ (similar to NEERC’18),  
    and connect $(u,e_{i,u}), (u',e_{i,u}), (v,e_{i,v}), (v',e_{i,v})$ with zero-weight edges.  
    A perfect matching exists $\Leftrightarrow$ a disjoint cycle cover exists.  
    After finding a maximum-weight matching, subtract the total matching weight from the sum of all edge weights to get the result.

    \item \textbf{Two Matching}:  
    Find a maximum-weight matching where each vertex can be incident to at most two edges.\\
    Each connected component must be either a single vertex, a path, or a cycle.  
    Add zero-weight edges between every pair of distinct vertices and also add a zero-weight $(v,v')$ edge — this turns the problem into the Disjoint Cycle Cover problem.  
    A component with a single vertex can be treated as having a self-loop;  
    for path components, it helps to think of connecting the two endpoints.
\end{itemize}
